% This is main.tex, a sample paper demonstrating the use of the
% LLNCS macro package for Springer Computer Science proceedings;
% Version 2.20 of 2017/10/04
% 
\documentclass[runningheads]{llncs}
%
% ---- Packages ----
%
\usepackage{graphicx} % enhanced support for graphics
\usepackage{url} % add macros for handling URLs in text
\usepackage[nohyperlinks,nolist]{acronym} % abbreviation utilities
\usepackage{listings}
% TODO: add more packages below if necessary
%
% ---- Acronyms ----
%
\begin{acronym}
\acro{rq}[RQ]{Research Question}
% TODO: define more acronyms here
\end{acronym}
%
% ---- Begin Document ----
%
\begin{document}
%
\title{Title of Seminar Paper}
%
%\titlerunning{Abbreviated paper title}
% If the paper title is too long for the running head, you can set
% an abbreviated paper title here
%
% ---- Author Information ----
%
\author{Jonas Schuhmacher, Jakob Rott}
\institute{Seminar: Software Quality (SS2022)\\
Technical University of Munich\\
\email{jonas.schuhmacher@tum.de, rott@cqse.eu}}
%
\maketitle % typeset the header of the contribution
%
% ---- Abstract ----
%
\begin{abstract}
The abstract should briefly summarize the contents of the paper in
15--250 words.

\keywords{First keyword  \and Second keyword \and Another keyword.}
\end{abstract}
%
% ---- Text Parts ----
%
% !TeX spellcheck = en_US

\section{Introduction}
\label{sec:intro}

What exactly does the word \textit{quality} mean? The Oxford Dictionary defines it as "the standard of something as measured against other things of a similar kind".\footnote{\url{https://arcadia.io/the-final-word-quality/} last accessed: 15.07.2022}, which only leads to more questions about the meaning of the words: standard, measurement, similar kind. A pretty tricky definition, although the word can be found nearly everywhere.
If one zooms in, one can try to partially define the word, e.g., in the context of software engineering. According to the ISO 9000 - Quality management, "(Software) Quality is the degree to which a set of inherent characteristics fulfills requirements, meaning needs or expectations that are stated, generally implied or obligatory" \cite{matthes2020ase}.
Even this restricted definition is still a bit fuzzy. By further zooming in, one can separate between internal and external quality. Here, the former describes the properties of the code in the context of a software system \cite{pretschner2022requirements}, whereas the latter formulates the user-observable criteria.
In the following, we will stay with the former and focus on the criterion clone coverage specifically.
This property can be defined as the density of code clones and describes the relative number of statements being part of a clone, i.e., duplicated \cite{knilling2020priorisierung}.
The phenomenon of code clones is a relatively common issue in software projects. Previous studies showed that software projects typically contain $7 -23\%$ code duplication \cite{koschke2007survey}.
However, there is no absolute consensus about the harmfulness of code clones, similar to the fuzzy definition of quality. As Kasper et al. \cite{kasper2006cloning} state, cloning can be an efficient engineering tool for testing new features, i.e., quick prototyping.
On the other hand, code duplicates can "increase maintenance costs [and] can create faults" \cite{juergens2009code} due to inconsistent changes to them, which harm consequently correctness and, thereby, quality. Generally, this work treats code clones neither as good nor bad but instead tries to understand quantitatively which effect the choice of programming language has on clone coverage.
\section{Conclusion}
\label{sec:conclusion}

You can also reference other parts of the document, e.g., sections or subsections.
In Section~\ref{sec:intro} we briefly introduced something, whereas in Subsection~\ref{sec:intro:sub:motivation}, we motivated something else.

Make sure to capitalize chapters, sections or subsections when referencing them.
%
% ---- Appendix ----
%
\appendix
\section{Appendix}
\label{sec:appendix}

Anything additional goes here \dots
%
% ---- Bibliography ----
%
\bibliographystyle{splncs04}
\bibliography{library}
%
\end{document}
%
% ---- Begin Document ----
%
