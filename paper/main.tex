% This is main.tex, a sample paper demonstrating the use of the
% LLNCS macro package for Springer Computer Science proceedings;
% Version 2.20 of 2017/10/04
% 
\documentclass[runningheads]{llncs}
%
% ---- Packages ----
%
\usepackage{graphicx} % enhanced support for graphics
\usepackage{url} % add macros for handling URLs in text
\usepackage[nohyperlinks,nolist]{acronym} % abbreviation utilities
\usepackage{listings}
\usepackage{hyperref}



% Capitalize autoref{..}
\def\chapterautorefname{Chapter}
\def\sectionautorefname{Section}
\def\subsectionautorefname{Subsection}
\def\algorithmautorefname{Algorithm}
\def\subfigureautorefname{Figure}
\def\itemautorefname{Step}
%
% ---- Acronyms ----
%
\begin{acronym}
\acro{rq}[RQ]{Research Question}
\end{acronym}
%
% ---- Begin Document ----
%
\begin{document}
%
\title{Evaluating Code Clone Frequency In Different-Generation Programming Languages}
%
%\titlerunning{Abbreviated paper title}
% If the paper title is too long for the running head, you can set
% an abbreviated paper title here
%
% ---- Author Information ----
%
\author{Jonas Schuhmacher, Jakob Rott}
\institute{Seminar: Software Quality (SS2022)\\
Technical University of Munich\\
\email{jonas.schuhmacher@tum.de, rott@cqse.eu}}
%
\maketitle % typeset the header of the contribution
%
% ---- Abstract ----
%
\begin{abstract}
The abstract should briefly summarize the contents of the paper in
15--250 words.

\keywords{First keyword  \and Second keyword \and Another keyword.}
\end{abstract}
%
% ---- Text Parts ----
%
% !TeX spellcheck = en_US

\section{Introduction}
\label{sec:intro}

What exactly does the word \textit{quality} mean? The Oxford Dictionary defines it as "the standard of something as measured against other things of a similar kind".\footnote{\url{https://arcadia.io/the-final-word-quality/} last accessed: 15.07.2022}, which only leads to more questions about the meaning of the words: standard, measurement, similar kind. A pretty tricky definition, although the word can be found nearly everywhere.
If one zooms in, one can try to partially define the word, e.g., in the context of software engineering. According to the ISO 9000 - Quality management, "(Software) Quality is the degree to which a set of inherent characteristics fulfills requirements, meaning needs or expectations that are stated, generally implied or obligatory" \cite{matthes2020ase}.
Even this restricted definition is still a bit fuzzy. By further zooming in, one can separate between internal and external quality. Here, the former describes the properties of the code in the context of a software system \cite{pretschner2022requirements}, whereas the latter formulates the user-observable criteria.
In the following, we will stay with the former and focus on the criterion clone coverage specifically.
This property can be defined as the density of code clones and describes the relative number of statements being part of a clone, i.e., duplicated \cite{knilling2020priorisierung}.
The phenomenon of code clones is a relatively common issue in software projects. Previous studies showed that software projects typically contain $7 -23\%$ code duplication \cite{koschke2007survey}.
However, there is no absolute consensus about the harmfulness of code clones, similar to the fuzzy definition of quality. As Kasper et al. \cite{kasper2006cloning} state, cloning can be an efficient engineering tool for testing new features, i.e., quick prototyping.
On the other hand, code duplicates can "increase maintenance costs [and] can create faults" \cite{juergens2009code} due to inconsistent changes to them, which harm consequently correctness and, thereby, quality. Generally, this work treats code clones neither as good nor bad but instead tries to understand quantitatively which effect the choice of programming language has on clone coverage.
% !TeX spellcheck = en_US

\section{Related Work}
\label{sec:related_work}

In general, we can subdivide this section into two categories. First, existing works searching for the reasons for code clones in human, organizational, and technological factors are examined, and secondly, we focus on papers describing the correlation between specific languages and the appearance of code clones explicitly.

\subsection{Reasons for Code Clones}
\label{sec:reasons_for_code_clones}

\subsubsection{Human \& Organizational Factors}

Before describing how the choice of programming language can affect clones, we start with a brief overview of other reasons for clones. Here, first to mention are human factors like inadvertently or impatiently copying because time or understanding are missing. Missing abstraction and the need to fulfill particular coupling and cohesion properties are also potential sources of generating code clones  \cite{kasper2006cloning}.
The intention to minimize the risks when adopting new ideas in software is also a driver for cloning code since this technique allows to keep errors through the introduced redundancy in just a single module  \cite{cordy2003comprehending}. Further, as time-to-market is critical, cloning can improve the speed of developing an early prototype, as analyzed by \cite{rajapakse2007using} for web applications.

\subsubsection{Technological Factors}

Alongside these human and economic factors stand "[t]echnology limitations" \cite{kasper2006cloning} like using a specific programming language.
\textit{Templating} is a powerful feature of modern programming languages like \texttt{C++}. In contrast, languages like \texttt{C} do not offer any equivalent feature.
Given a certain algorithm working with double precision in the \texttt{C} programming language, the developer is forced to copy the procedure and replace \texttt{double} with \texttt{float}. Such "boiler-plating [is enforced only] due to language inexpressiveness" \cite{kasper2006cloning}. 
Besides those issues with \textit{Templating}, i.e., generalization, in specific languages exist deadlocked ways and patterns how to fulfill specific tasks.
For example, creating a \textit{Socket} and communicating with it varies in each programming language - sometimes shorter, sometimes more lengthy. Nonetheless, there is usually one optimal way of creating it and handling possible exceptions (if the language even has such a feature) or error codes. 
These optimal sequences will then be often copy-and-pasted by developers \cite{kasper2006cloning}.
Further to \textit{Templating}, Kasper et al.  \cite{kasper2006cloning} mention \textit{customization} as a reason for duplication.
For instance, code ownership can make bug fixing difficult, only allowing the creation of a work-around by copying and improving the faulty lines.
Further, they enumerate the idiom of \textit{replicate and specialize}, in which developers clone code to specialize a solution rather than generalizing an existing implementation.
Lastly, Kasper et al. \cite{kasper2006cloning} describe \textit{exact matches}, in the context of code clones, as a result of copying ”semantic properties [of] otherwise unrelated functionality” \cite{kasper2006cloning} between methods like logging or debugging statements and reusing characteristic code structures like loops which are easier copied than implemented as a reusable function.

\subsection{Similar Analyses}
\label{sec:similiar_analyses}

This work will later compare clone coverage in different generation programming languages. Other studies with a similar scope and a somewhat different focus are concerned in this subsection.

\subsubsection{Code Clones in Build Systems}

Once a developer has produced the different components of a large-scale software system, they typically need to be assembled. This critical step is usually accomplished with build systems like \texttt{Ant},  \texttt{Maven}, \texttt{Autotools}, or \texttt{CMake} for respectively \texttt{Java} and \texttt{C/C++}. McIntosh et al. \cite{mcintosh2014collecting} compares these build systems with surprising results. Overall, "build systems tend to have higher cloning rates than other software artifacts" \cite{mcintosh2014collecting}. They conclude that, as opposed to what one might think, modern build systems like \texttt{CMake} and  \texttt{Maven} have higher clone coverage than their older counterparts. Moreover, the Java build systems have a higher clone coverage across-the-board than their \texttt{C/C++} companions - in many cases, \texttt{Java} build systems reach more than $50\%$ clone coverage.

This paper relates to the procedure here as build systems are technological choices similar to programming language selection, and the right choice might reduce clones.

\subsubsection{Comparison of Java \& Scala}

Jorge et al. \cite{jorge2012impact} directly compare features of the two languages, \texttt{Java} and \texttt{Scala}, and study how their language constructs correlate to code cloning. Their findings conclude that code duplication problems arise with a higher probability if a language is verbose, e.g., due to getters and setters, anonymous classes, constructors, and lack (simple) abstraction capabilities. Properties which finally lead to more effort refactoring code than simply cloning a source code. \cite{jorge2012impact}
Their approach is more qualitative than the presented quantitative method but can be seen as a great inspiration for this work.

\section{Methodology}
\label{sec:methodology}

Present the UML structure of the analysis, as well as the selection criterion/ sources.
\section{Results}
\label{sec:results}

Present the plots and give some rough description of on them
\section{Discussion}
\label{sec:discussion}

Confirm the thesis.
\section{Conclusion}
\label{sec:conclusion}

Summarize and give outlook

%
% ---- Appendix ----
%
\appendix
\section{Appendix}
\label{sec:appendix}

Anything additional goes here \dots
%
% ---- Bibliography ----
%
\bibliographystyle{splncs04}
\bibliography{library}
%
\end{document}
%
% ---- Begin Document ----
%
