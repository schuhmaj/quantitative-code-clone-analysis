% !TeX spellcheck = en_US
% This is main.tex, a sample paper demonstrating the use of the
% LLNCS macro package for Springer Computer Science proceedings;
% Version 2.20 of 2017/10/04
% 
\documentclass[runningheads]{llncs}
%
% ---- Packages ----
%
\usepackage{graphicx} % enhanced support for graphics
\usepackage{url} % add macros for handling URLs in text
\usepackage[nohyperlinks,nolist]{acronym} % abbreviation utilities
\usepackage{listings}
\usepackage{hyperref}



% Capitalize autoref{..}
\def\chapterautorefname{Chapter}
\def\sectionautorefname{Section}
\def\subsectionautorefname{Subsection}
\def\algorithmautorefname{Algorithm}
\def\subfigureautorefname{Figure}
\def\itemautorefname{Step}
%
% ---- Acronyms ----
%
\begin{acronym}
\acro{rq}[RQ]{Research Question}
\end{acronym}
%
% ---- Begin Document ----
%
\begin{document}
%
\title{Evaluating Code Clone Frequency In Different-Generation Programming Languages}
%
%\titlerunning{Abbreviated paper title}
% If the paper title is too long for the running head, you can set
% an abbreviated paper title here
%
% ---- Author Information ----
%
\author{Jonas Schuhmacher}
\institute{Seminar: Software Quality (SS2022)\\
Technical University of Munich\\
\email{jonas.schuhmacher@tum.de}}
%
\maketitle % typeset the header of the contribution
%
% ---- Abstract ----
%
\begin{abstract}
The abstract should briefly summarize the contents of the paper in
15--250 words.
%TODO
\keywords{Code Clone  \and Frequency \and Programming Language.}
\end{abstract}
%
% ---- Text Parts ----
%
% !TeX spellcheck = en_US

\section{Introduction}
\label{sec:intro}

What exactly does the word \textit{quality} mean? The Oxford Dictionary defines it as "the standard of something as measured against other things of a similar kind".\footnote{\url{https://arcadia.io/the-final-word-quality/} last accessed: 15.07.2022}, which only leads to more questions about the meaning of the words: standard, measurement, similar kind. A pretty tricky definition, although the word can be found nearly everywhere.
If one zooms in, one can try to partially define the word, e.g., in the context of software engineering. According to the ISO 9000 - Quality management, "(Software) Quality is the degree to which a set of inherent characteristics fulfills requirements, meaning needs or expectations that are stated, generally implied or obligatory" \cite{matthes2020ase}.
Even this restricted definition is still a bit fuzzy. By further zooming in, one can separate between internal and external quality. Here, the former describes the properties of the code in the context of a software system \cite{pretschner2022requirements}, whereas the latter formulates the user-observable criteria.
In the following, we will stay with the former and focus on the criterion clone coverage specifically.
This property can be defined as the density of code clones and describes the relative number of statements being part of a clone, i.e., duplicated \cite{knilling2020priorisierung}.
The phenomenon of code clones is a relatively common issue in software projects. Previous studies showed that software projects typically contain $7 -23\%$ code duplication \cite{koschke2007survey}.
However, there is no absolute consensus about the harmfulness of code clones, similar to the fuzzy definition of quality. As Kasper et al. \cite{kasper2006cloning} state, cloning can be an efficient engineering tool for testing new features, i.e., quick prototyping.
On the other hand, code duplicates can "increase maintenance costs [and] can create faults" \cite{juergens2009code} due to inconsistent changes to them, which harm consequently correctness and, thereby, quality. Generally, this work treats code clones neither as good nor bad but instead tries to understand quantitatively which effect the choice of programming language has on clone coverage.
% !TeX spellcheck = en_US

\section{Related Work}
\label{sec:related_work}

In general, we can subdivide this section into two categories. First, existing works searching for the reasons for code clones in human, organizational, and technological factors are examined, and secondly, we focus on papers describing the correlation between specific languages and the appearance of code clones explicitly.

\subsection{Reasons for Code Clones}
\label{sec:reasons_for_code_clones}

\subsubsection{Human \& Organizational Factors}

Before describing how the choice of programming language can affect clones, we start with a brief overview of other reasons for clones. Here, first to mention are human factors like inadvertently or impatiently copying because time or understanding are missing. Missing abstraction and the need to fulfill particular coupling and cohesion properties are also potential sources of generating code clones  \cite{kasper2006cloning}.
The intention to minimize the risks when adopting new ideas in software is also a driver for cloning code since this technique allows to keep errors through the introduced redundancy in just a single module  \cite{cordy2003comprehending}. Further, as time-to-market is critical, cloning can improve the speed of developing an early prototype, as analyzed by \cite{rajapakse2007using} for web applications.

\subsubsection{Technological Factors}

Alongside these human and economic factors stand "[t]echnology limitations" \cite{kasper2006cloning} like using a specific programming language.
\textit{Templating} is a powerful feature of modern programming languages like \texttt{C++}. In contrast, languages like \texttt{C} do not offer any equivalent feature.
Given a certain algorithm working with double precision in the \texttt{C} programming language, the developer is forced to copy the procedure and replace \texttt{double} with \texttt{float}. Such "boiler-plating [is enforced only] due to language inexpressiveness" \cite{kasper2006cloning}. 
Besides those issues with \textit{Templating}, i.e., generalization, in specific languages exist deadlocked ways and patterns how to fulfill specific tasks.
For example, creating a \textit{Socket} and communicating with it varies in each programming language - sometimes shorter, sometimes more lengthy. Nonetheless, there is usually one optimal way of creating it and handling possible exceptions (if the language even has such a feature) or error codes. 
These optimal sequences will then be often copy-and-pasted by developers \cite{kasper2006cloning}.
Further to \textit{Templating}, Kasper et al.  \cite{kasper2006cloning} mention \textit{customization} as a reason for duplication.
For instance, code ownership can make bug fixing difficult, only allowing the creation of a work-around by copying and improving the faulty lines.
Further, they enumerate the idiom of \textit{replicate and specialize}, in which developers clone code to specialize a solution rather than generalizing an existing implementation.
Lastly, Kasper et al. \cite{kasper2006cloning} describe \textit{exact matches}, in the context of code clones, as a result of copying ”semantic properties [of] otherwise unrelated functionality” \cite{kasper2006cloning} between methods like logging or debugging statements and reusing characteristic code structures like loops which are easier copied than implemented as a reusable function.

\subsection{Similar Analyses}
\label{sec:similiar_analyses}

This work will later compare clone coverage in different generation programming languages. Other studies with a similar scope and a somewhat different focus are concerned in this subsection.

\subsubsection{Code Clones in Build Systems}

Once a developer has produced the different components of a large-scale software system, they typically need to be assembled. This critical step is usually accomplished with build systems like \texttt{Ant},  \texttt{Maven}, \texttt{Autotools}, or \texttt{CMake} for respectively \texttt{Java} and \texttt{C/C++}. McIntosh et al. \cite{mcintosh2014collecting} compares these build systems with surprising results. Overall, "build systems tend to have higher cloning rates than other software artifacts" \cite{mcintosh2014collecting}. They conclude that, as opposed to what one might think, modern build systems like \texttt{CMake} and  \texttt{Maven} have higher clone coverage than their older counterparts. Moreover, the Java build systems have a higher clone coverage across-the-board than their \texttt{C/C++} companions - in many cases, \texttt{Java} build systems reach more than $50\%$ clone coverage.

This paper relates to the procedure here as build systems are technological choices similar to programming language selection, and the right choice might reduce clones.

\subsubsection{Comparison of Java \& Scala}

Jorge et al. \cite{jorge2012impact} directly compare features of the two languages, \texttt{Java} and \texttt{Scala}, and study how their language constructs correlate to code cloning. Their findings conclude that code duplication problems arise with a higher probability if a language is verbose, e.g., due to getters and setters, anonymous classes, constructors, and lack (simple) abstraction capabilities. Properties which finally lead to more effort refactoring code than simply cloning a source code. \cite{jorge2012impact}
Their approach is more qualitative than the presented quantitative method but can be seen as a great inspiration for this work.

% !TeX spellcheck = en_US

\section{Research Question}
\label{sec:research_question}

\autoref{sec:reasons_for_code_clones} presented reasons how the choice of programming language can affect the number of code clones in a software project. \autoref{sec:similiar_analyses} showed recent findings in terms of a comparison between build systems and an actual fine-granular examination of \texttt{Java} and \texttt{Scala}.
This paper builds on top of those results and examines the correlation of clone coverage in different programming languages in a quantitatively large-scale. Hereby, the following \ac{rq} are discussed on the remaining pages:

\begin{enumerate}
	\item Which programming languages suffer mostly from high clone coverage?
	\item Do modern programming languages have lower clone coverage compared to older languages?
	\begin{enumerate}
		\item Does pure \texttt{C} generally have a higher clone coverage than \texttt{C/C++}?
		\item Does \texttt{Rust} improve \texttt{C/C++} in terms of clone coverage?
		\item Does \texttt{Kotlin} improve \texttt{Java} in terms of clone coverage?
	\end{enumerate}
	\item Can we infer that a simpler, less verbose, more modern programming language always leads to less code clones?
\end{enumerate}

Important to note is that a question analyzing the "Why?" is missing since the following case study relies on quantitatively measuring code clone coverage in thousands of projects by just examining the source code, but not concrete software engineering process, i.e. principles of those projects. Rather \autoref{sec:results} will show certain trends for specific programming languages which are then discussed based on the formulated research questions in \autoref{sec:discussion}.
% !TeX spellcheck = en_US

\section{Study Design}
\label{sec:study_design}


\subsection{Overview}

\subsection{Implementation}
% !TeX spellcheck = en_US

\section{Results}
\label{sec:results}

This section now provides the results and a brief description of the previously presented study design. 

\begin{figure}[tbh!]
	\centering
	\includegraphics[width=0.85\linewidth]{figures/results/scatter_clone_coverage_loc_m}
	\caption{Clone Coverage depending on projects' \ac{sloc} with trend lines plotted}
	\label{fig:overview_results}
\end{figure}

\begin{figure}[tbh!]
	\centering
	\includegraphics[width=0.85\linewidth]{figures/results/number}
	\caption{Complementary cumulative distribution function for the number of projects given a certain \ac{sloc} threshold}
	\label{fig:overview_numbers}
\end{figure}

A first overview is given in \autoref{fig:overview_results}, which plots the clone coverage in dependence to the project size, i.e. the \acl{sloc}, for all analyzed projects. Further, the plot contains a trend line for each programming language in order to enable better comparability. These trend lines depict the mean value of clone coverage for every programming language until around $10^6$ of \ac{sloc} pretty close. Also, they clearly show that bigger projects tend to have a greater clone coverage.
Nevertheless, the data far beyond $10^6$ \acl{sloc} should be taken with caution for some programming languages since not enough data could be collected for the bigger size regimes. 
\autoref{fig:overview_numbers} portrays this situation with the tick $10^6$ \ac{sloc} being marked since before $10^6$, we have at least 100 projects for every programming language, whereas languages like \texttt{Kotlin} and \texttt{Rust} only have small amounts of samples for e.g. more than $10^6$ \ac{sloc}.

\autoref{fig:histo_all} extends the information of \autoref{fig:overview_results} by giving the exact distributions, as well as, the mean and standard deviation values for every programming language. If one compare the mean clone coverage values for every programming language with one another, modern programming languages like \texttt{Rust}, \texttt{Kotlin} and \texttt{Go} seem to perform better than their older counterparts \texttt{C/C++} and \texttt{Java}. Further, \texttt{C} and \texttt{C++} appear to have only little difference in terms of clone coverage distributions. \texttt{Python} performs overall best.

Supplementary to the total distribution, \autoref{fig:histo_million} narrows this down to just presenting projects with more than one million \acl{sloc}. As previously mentioned, one million \ac{sloc} is the range where still enough data is present for some adequate evidence. These projects are also more expressive due to their sheer size. The general observations still holds, with one exception: \texttt{Go} performs far worse. Further as already perceived in \autoref{fig:overview_results}, all languages have greater mean values overall.

Since the pure observation is not sufficient for answering the \aclp{rq} precisely, the two-sample Welch's $t$-test is used to determine if the differences in clone coverage mean values are significant. This test assumes that the sample mean values are distributed normally which holds due to the Central Limit Theorem \cite{shafer2022introductory}. In contrast to Student's $t$-test it allows unequal variances, which exist here. \autoref{tab:stat_test} shows the calculated $t$ statistics and $p$ values for four scenarios. With a level of significance of $\alpha = 5\%$, \texttt{C} and \texttt{C/C++} mean values are not significantly different since $p > \alpha$ leads to acceptance of $H_0$: equal means. This circumstance answers \refrq{question:compare_c_cpp} with no.

The next two rows answer \refrq{question:compare_rust_c} and \refrq{question:compare_kotlin_java}. In both cases are the mean values significantly different, which leads to rejection of $H_0$ and acceptance of $H_1$: \texttt{C/C++} and \texttt{Java} have on average a significantly higher mean clone coverage than respectively \texttt{Rust} and \texttt{Kotlin}. So both questions can be answered with "yes".
Lastly, \texttt{C/C++} and \texttt{Java} are compared with the result of no significant difference in the mean values for $\alpha = 5\%$.

\begin{table}[tbh!]
	\centering
	\begin{tabular}{|cc||c|c||c|c|}
		\hline
		\multicolumn{2}{|c||}{Samples} & \multicolumn{2}{c||}{$SLOC \geq 0$} & \multicolumn{2}{c|}{$SLOC \geq 1000000$}  \\
		\multicolumn{2}{|c||}{from} & \multicolumn{1}{c}{$t$ statistic} & \multicolumn{1}{c||}{$p$ value} & \multicolumn{1}{c}{$t$ statistic} & $p$ value \\
		\hline
		\hline
		C & C/C++ & $-1.021$ & $0.846$ & $1.105$ & $0.135$ \\
		\hline
		\hline
		C/C++ & Rust & $9.936$ & $1.517 \cdot 10^{-22}$ & $6.612$ & $3.957 \cdot 10^{-11}$ \\
		\hline
		Java & Kotlin & $5.946$ & $2.264 \cdot 10^{-9}$ & $2.500$ & $0.006$ \\
		\hline
		\hline
		C/C++ & Java & $0.720$ & $0.236$ & $0.643$ & $0.260$ \\
		\hline
	\end{tabular}
	\caption{Results of Welch's $t$-test with $H_0$ being that the means of the two underlying distributions have equal means and $H_1$ that the mean of the first distribution is greater than the mean of the second sample.}
	\label{tab:stat_test}
\end{table}

Overall, \autoref{fig:matrix_comp} lists all possible pairings of examined programming languages and depicts which one performs significantly better than its companion. \autoref{fig:matrix_comp} complements \refrq{question:comparing_age} by giving answer to the top-level \ac{rq} and gives a clear answer for \refrq{question:clone_coverage_lowest}.
\autoref{fig:matrix_comp} clearly depicts that \texttt{Python} always performs better in terms of mean code clone coverage than any other programming language independent from the applied filter for \aclp{sloc}. So  \refrq{question:clone_coverage_lowest}'s answer is \texttt{Python}, which was first released in 1991\footnote{\url{https://en.wikipedia.org/wiki/Python_(programming_language)}, lasted accessed: 13.07.2022}.
This fact also implies that newer generation programming languages do not necessarily have lower code clone coverage than older programming languages. The fact that \texttt{Go} behaves in terms of mean code clone coverage similarly to \texttt{C/C++} and \texttt{Java} supports this circumstance, even if it first appeared in 2009\footnote{\url{https://en.wikipedia.org/wiki/Go_(programming_language)}, last accessed: 13.07.2022}.

However, fully answering \refrq{question:comparing_age} with no, would be short-sighted, as the direct competitors \texttt{Rust} and \texttt{Kotlin} perform better than \texttt{C/C++} and \texttt{Java}. These discoveries are broadly discussed in \autoref{sec:discussion}.

\begin{figure}[tbh!]
	\centering
	\begin{subfigure}[t]{0.49\textwidth}
		\includegraphics[width=\textwidth]{figures/comparison/comparison_all_5percent}
		\caption{with minimal $SLOC > 0$}
		\label{fig:matrix_comp_all}
	\end{subfigure}
	\hfill
	\begin{subfigure}[t]{0.49\textwidth}
		\includegraphics[width=\textwidth]{figures/comparison/comparison_million_5percent}
		\caption{with minimal $SLOC \geq 10^{6}$}
		\label{fig:matrix_comp_million}
	\end{subfigure}
	\caption{Matrix comparison which programming languages has significantly lower mean clone coverage (green), equal mean clone coverage (yellow) and higher mean clone coverage (red) with an $\alpha=5\%$. The comparison is performed by using Welch's $t$-tests. The matrix is read in the following fashion: "Sample 1 performs better/neutral/worse than Sample 2".}
	\label{fig:matrix_comp}
\end{figure}

\begin{figure}[p]
	\centering
	\begin{subfigure}[t]{0.49\textwidth}
		\includegraphics[width=\textwidth]{figures/results/c/histogram_all}
		\caption{for pure \texttt{C}}
		\label{fig:histo_all_c}
	\end{subfigure}
	\hfill
	\begin{subfigure}[t]{0.49\textwidth}
		\includegraphics[width=\textwidth]{figures/results/cpp/histogram_all}
		\caption{for \texttt{C/C++}}
		\label{fig:histo_all_cpp}
	\end{subfigure}
	\begin{subfigure}[t]{0.49\textwidth}
		\includegraphics[width=\textwidth]{figures/results/rust/histogram_all}
		\caption{for \texttt{Rust}}
		\label{fig:histo_all_rust}
	\end{subfigure}
	\begin{subfigure}[t]{0.49\textwidth}
		\includegraphics[width=\textwidth]{figures/results/python/histogram_all}
		\caption{for \texttt{Python}}
		\label{fig:histo_all_python}
	\end{subfigure}
	\begin{subfigure}[t]{0.49\textwidth}
		\includegraphics[width=\textwidth]{figures/results/java/histogram_all}
		\caption{for \texttt{Java}}
		\label{fig:histo_all_java}
	\end{subfigure}
	\begin{subfigure}[t]{0.49\textwidth}
		\includegraphics[width=\textwidth]{figures/results/kotlin/histogram_all}
		\caption{for \texttt{Kotlin}}
		\label{fig:histo_all_kotlin}
	\end{subfigure}
	\begin{subfigure}[t]{0.49\textwidth}
		\includegraphics[width=\textwidth]{figures/results/go/histogram_all}
		\caption{for \texttt{Go}}
		\label{fig:histo_all_go}
	\end{subfigure}
	\caption{Histograms of the clone coverage distribution for the seven examined programming languages containing all project. Each plot contains the mean value and the standard deviation.}
	\label{fig:histo_all}
\end{figure}


\begin{figure}[p]
	\centering
	\begin{subfigure}[t]{0.49\textwidth}
		\includegraphics[width=\textwidth]{figures/results/c/histogram_million}
		\caption{for pure \texttt{C}}
		\label{fig:histo_million_c}
	\end{subfigure}
	\hfill
	\begin{subfigure}[t]{0.49\textwidth}
		\includegraphics[width=\textwidth]{figures/results/cpp/histogram_million}
		\caption{for \texttt{C/C++}}
		\label{fig:histo_million_cpp}
	\end{subfigure}
	\begin{subfigure}[t]{0.49\textwidth}
		\includegraphics[width=\textwidth]{figures/results/rust/histogram_million}
		\caption{for \texttt{Rust}}
		\label{fig:histo_million_rust}
	\end{subfigure}
	\begin{subfigure}[t]{0.49\textwidth}
		\includegraphics[width=\textwidth]{figures/results/python/histogram_million}
		\caption{for \texttt{Python}}
		\label{fig:histo_million_python}
	\end{subfigure}
	\begin{subfigure}[t]{0.49\textwidth}
		\includegraphics[width=\textwidth]{figures/results/java/histogram_million}
		\caption{for \texttt{Java}}
		\label{fig:histo_million_java}
	\end{subfigure}
	\begin{subfigure}[t]{0.49\textwidth}
		\includegraphics[width=\textwidth]{figures/results/kotlin/histogram_million}
		\caption{for \texttt{Kotlin}}
		\label{fig:histo_million_kotlin}
	\end{subfigure}
	\begin{subfigure}[t]{0.49\textwidth}
		\includegraphics[width=\textwidth]{figures/results/go/histogram_million}
		\caption{for \texttt{Go}}
		\label{fig:histo_million_go}
	\end{subfigure}
	\caption{Histograms of the clone coverage distribution for the seven examined programming languages containing only projects with at least one million \ac{sloc}. Each plot contains the mean value and the standard deviation.}
	\label{fig:histo_million}
\end{figure}
% !TeX spellcheck = en_US

\section{Discussion}
\label{sec:discussion}

\autoref{sec:results} has shown that the in \autoref{sec:study_design} presented approach for quantitatively measuring clone coverage in different programming languages is capable to answer some of the asked \aclp{rq}. This section is going to discuss and explain the observed results, but will also evaluate the used approach.

\subsection{Implications of the result}

%Can we infer that a simpler, less verbose, more modern programming language always leads to less code clones?

Generally, the extracted data does not imply that a certain programming language like e.g. \texttt{Python}, which performed best, should be used for every project. However, the results show that certain languages, \texttt{Python}, \texttt{Kotlin} and \texttt{Rust}, performed significantly better than \texttt{C/C++}, \texttt{Java}, and therefore, seem to have advantages towards the other ones.
One reason for fewer clones could be a less-verbose, simpler programming language design as discussed in \autoref{sec:similiar_analyses}, yet this argument can be discarded, since \texttt{Go} performs for projects bigger than $10^6$ \ac{sloc} similar to \texttt{Java} and \texttt{C/C++}. Although \texttt{Go} has the smallest set of keywords - implying simplicity, as depicted in \autoref{tab:keyword_number}.

\begin{table}[tbh!]
	\centering
	\begin{tabular}{|>{\centering\arraybackslash}m{2cm}|>{\centering\arraybackslash}m{1.5cm}|>{\centering\arraybackslash}m{1.5cm}|>{\centering\arraybackslash}m{1.5cm}|>{\centering\arraybackslash}m{1.5cm}|>{\centering\arraybackslash}m{2cm}|>{\centering\arraybackslash}m{1.5cm}|}
		\hline
		C/C++ 17 & Kotlin 1.4 & Rust 1.46 & Java 14 & C18 & Python 3.8 & Go 1.15 \\
		\hline
		84 & 79 & 53 & 51 & 44 & 35 & 25 \\
		\hline
	\end{tabular}
	\caption{Number of keywords for each examined programming language \cite{meyer2022keywords}}
	\label{tab:keyword_number}
\end{table}

Further, \autoref{sec:similiar_analyses} already introduced build systems as crucial for projects success. Here might be another reason located. A simple to use package manager makes importing existing solutions much easier, thereby prevents "Reinventing the wheel" a.k.a copying existing solutions. \texttt{Rust} and \texttt{Python} both come with package managers \texttt{cargo} and \texttt{pip} making the use of existing solutions much easier. In comparison, \texttt{C/C++} and \texttt{Java} rely on build systems like \texttt{CMake} and \texttt{Maven}, usually with a high learning curve.

Lastly, one one does not forget that e.g. \texttt{Kotlin} and \texttt{Rust} improve \texttt{Java} and \texttt{C/C++}, respectively introducing null-safety and ownership. Thus these improvements avoid redundant often repeated runtime-checks like mentioned in \autoref{sec:reasons_for_code_clones} (reusing certain code structures).

In spite of these reasons, the difference of mean clone coverage do not necessarily indicate that the programming language has flaws.
Older projects often has a greater (legacy) code base, which already can contain clones, and makes it easy to just copy and paste from there. However, projects in \texttt{Rust}, \texttt{Kotlin} and \texttt{Go} are not yet as old as projects in \texttt{C/C++} or \texttt{Java}. Thereby, the newer programming languages do have some "time" advantage towards the older ones.

 
\subsection{Threads to validity}

Clone Coverage has previously been introduced as the percentage of code lines covered by at least one clone. Based on this definition, one can also refer to clone coverage as the probability that a random change to a line of code across the project implies a propagation to another line being a clone.
 

% !TeX spellcheck = en_US

\section{Conclusion}
\label{sec:conclusion}

Summarize and give outlook

%
% ---- Appendix ----
%
\appendix
\section{Appendix}
\label{sec:appendix}

Anything additional goes here \dots
%
% ---- Bibliography ----
%
\bibliographystyle{splncs04}
\bibliography{library}
%
\end{document}
%
% ---- Begin Document ----
%
