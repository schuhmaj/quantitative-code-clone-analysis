% !TeX spellcheck = en_US

\section{Results}
\label{sec:results}

This section now provides the results and a brief description of the previously presented study design. 

\begin{figure}[tbh!]
	\centering
	\includegraphics[width=0.85\linewidth]{figures/results/scatter_clone_coverage_loc_m}
	\caption{Clone Coverage depending on projects' \ac{sloc} with trend lines plotted}
	\label{fig:overview_results}
\end{figure}

\begin{figure}[tbh!]
	\centering
	\includegraphics[width=0.85\linewidth]{figures/results/number}
	\caption{Complementary cumulative distribution function for the number of projects given a certain \ac{sloc} threshold}
	\label{fig:overview_numbers}
\end{figure}

A first overview is given in \autoref{fig:overview_results}, which plots the clone coverage in dependence to the project size, i.e. the \acl{sloc}, for all analyzed projects. Further, the plot contains a trend line for each programming language in order to enable better comparability. These trend lines depict the mean value of clone coverage for every programming language until around $10^6$ of \ac{sloc} pretty close. Also, they clearly show that bigger projects tend to have a greater clone coverage.
Nevertheless, the data far beyond $10^6$ \acl{sloc} should be taken with caution for some programming languages since not enough data could be collected for the bigger size regimes. 
\autoref{fig:overview_numbers} portrays this situation with the tick $10^6$ \ac{sloc} being marked since before $10^6$, we have at least 100 projects for every programming language, whereas languages like \texttt{Kotlin} and \texttt{Rust} only have small amounts of samples for e.g. more than $10^7$ \ac{sloc}.


\autoref{fig:histo_all} extends the information of \autoref{fig:overview_results} by giving the exact distributions, as well as, the mean and standard deviation values for every programming language. Supplementary, \autoref{fig:histo_million} narrows this down to just presenting projects with more than one million \acl{sloc}. As previously mentioned, one million \ac{sloc} is the range where still enough data is present for some adequate evidence. These projects are also more expressive due to their sheer size.

\begin{table}[tbh!]
	\centering
	\begin{tabular}{|cc||c|c||c|c|}
		\hline
		\multicolumn{2}{|c|}{Samples} & \multicolumn{2}{|c|}{$SLOC \geq 0$} & \multicolumn{2}{|c|}{$SLOC \geq 1000000$}  \\
		\multicolumn{2}{|c|}{from} & \multicolumn{1}{c}{$t$ statistic} & \multicolumn{1}{c|}{$p$ value} & \multicolumn{1}{c}{$t$ statistic} & $p$ value \\
		\hline
		\hline
		C & C/C++ & $-1.021$ & $0.846$ & $1.105$ & $0.135$ \\
		\hline
		\hline
		C/C++ & Rust & $9.936$ & $1.517 \cdot 10^{-22}$ & $6.612$ & $3.957 \cdot 10^{-11}$ \\
		\hline
		Java & Kotlin & $5.946$ & $2.264 \cdot 10^{-9}$ & $2.500$ & $0.006$ \\
		\hline
		\hline
		C/C++ & Java & $0.720$ & $0.236$ & $0.643$ & $0.260$ \\
		\hline
	\end{tabular}
	\caption{Results of two sample T-tests with $H_0$ being that the means of the two underlying distributions are equal and $H_1$ that the mean of the first distribution is greater than the mean of the second sample.}
	\label{tab:stat_test}
\end{table}

\begin{figure}[p]
	\centering
	\begin{subfigure}[t]{0.49\textwidth}
		\includegraphics[width=\textwidth]{figures/results/c/histogram_all}
		\caption{for pure \texttt{C}}
		\label{fig:histo_all_c}
	\end{subfigure}
	\hfill
	\begin{subfigure}[t]{0.49\textwidth}
		\includegraphics[width=\textwidth]{figures/results/cpp/histogram_all}
		\caption{for \texttt{C/C++}}
		\label{fig:histo_all_cpp}
	\end{subfigure}
	\begin{subfigure}[t]{0.49\textwidth}
		\includegraphics[width=\textwidth]{figures/results/rust/histogram_all}
		\caption{for \texttt{Rust}}
		\label{fig:histo_all_rust}
	\end{subfigure}
	\begin{subfigure}[t]{0.49\textwidth}
		\includegraphics[width=\textwidth]{figures/results/python/histogram_all}
		\caption{for \texttt{Python}}
		\label{fig:histo_all_python}
	\end{subfigure}
	\begin{subfigure}[t]{0.49\textwidth}
		\includegraphics[width=\textwidth]{figures/results/java/histogram_all}
		\caption{for \texttt{Java}}
		\label{fig:histo_all_java}
	\end{subfigure}
	\begin{subfigure}[t]{0.49\textwidth}
		\includegraphics[width=\textwidth]{figures/results/kotlin/histogram_all}
		\caption{for \texttt{Kotlin}}
		\label{fig:histo_all_kotlin}
	\end{subfigure}
	\begin{subfigure}[t]{0.49\textwidth}
		\includegraphics[width=\textwidth]{figures/results/go/histogram_all}
		\caption{for \texttt{Go}}
		\label{fig:histo_all_go}
	\end{subfigure}
	\caption{Histograms of the clone coverage distribution for the seven examined programming languages containing all project. Each plot contains the mean value and the standard deviation.}
	\label{fig:histo_all}
\end{figure}


\begin{figure}[p]
	\centering
	\begin{subfigure}[t]{0.49\textwidth}
		\includegraphics[width=\textwidth]{figures/results/c/histogram_million}
		\caption{for pure \texttt{C}}
		\label{fig:histo_million_c}
	\end{subfigure}
	\hfill
	\begin{subfigure}[t]{0.49\textwidth}
		\includegraphics[width=\textwidth]{figures/results/cpp/histogram_million}
		\caption{for \texttt{C/C++}}
		\label{fig:histo_million_cpp}
	\end{subfigure}
	\begin{subfigure}[t]{0.49\textwidth}
		\includegraphics[width=\textwidth]{figures/results/rust/histogram_million}
		\caption{for \texttt{Rust}}
		\label{fig:histo_million_rust}
	\end{subfigure}
	\begin{subfigure}[t]{0.49\textwidth}
		\includegraphics[width=\textwidth]{figures/results/python/histogram_million}
		\caption{for \texttt{Python}}
		\label{fig:histo_million_python}
	\end{subfigure}
	\begin{subfigure}[t]{0.49\textwidth}
		\includegraphics[width=\textwidth]{figures/results/java/histogram_million}
		\caption{for \texttt{Java}}
		\label{fig:histo_million_java}
	\end{subfigure}
	\begin{subfigure}[t]{0.49\textwidth}
		\includegraphics[width=\textwidth]{figures/results/kotlin/histogram_million}
		\caption{for \texttt{Kotlin}}
		\label{fig:histo_million_kotlin}
	\end{subfigure}
	\begin{subfigure}[t]{0.49\textwidth}
		\includegraphics[width=\textwidth]{figures/results/go/histogram_million}
		\caption{for \texttt{Go}}
		\label{fig:histo_million_go}
	\end{subfigure}
	\caption{Histograms of the clone coverage distribution for the seven examined programming languages containing only projects with at least one million \ac{sloc}. Each plot contains the mean value and the standard deviation.}
	\label{fig:histo_million}
\end{figure}