% !TeX spellcheck = en_US

\section{Research Questions}
\label{sec:research_question}

\autoref{sec:reasons_for_code_clones} presented reasons how the choice of programming language can affect the number of code clones in a software project. \autoref{sec:similiar_analyses} showed recent findings for clone distribution regarding build systems and an actual fine-granular examination of \texttt{Java} and \texttt{Scala}. Now, this paper examines the correlation of clone coverage in different programming languages on a quantitatively large scale. Hence, the following \ac{rq} are discussed on the remaining pages:

\bigskip
\begin{minipage}{0.95\linewidth}
	\begin{enumerate}[label=RQ \arabic*]
		\item Which of the seven examined programming language has the lowest (mean) clone coverage? \label{question:clone_coverage_lowest}
		\item Do modern programming languages have lower (mean) clone coverage than older ones? \label{question:comparing_age}
		\begin{enumerate}
			\item Does pure \texttt{C} generally have a higher clone coverage than \texttt{C/C++}? \label{question:compare_c_cpp}
			\item Does \texttt{Rust} outperform \texttt{C/C++} in terms of clone coverage? \label{question:compare_rust_c}
			\item Does \texttt{Kotlin} outperform \texttt{Java} in terms of clone coverage? \label{question:compare_kotlin_java}
		\end{enumerate}
	\end{enumerate}
\end{minipage}
\bigskip

\refrq{question:clone_coverage_lowest} searches for the most promising programming language in terms of code clone coverage and comes to the generalization if modern programming languages, released around 2010, are generally better than older languages from the last millennial in terms of code clone coverage.
\refrq{question:compare_c_cpp} is generally of interest because \texttt{C++} is a super-set of \texttt{C}. \refrq{question:compare_rust_c}  and \refrq{question:compare_kotlin_java} are two shapings of the same question if the more recent programming languages that aim for improving the shortcomings of previous generation languages are also more supportive to write less redundant code.

It is essential to note that no question explores the underlying reasons for code duplication since the following case study relies on quantitatively measuring code clone coverage in thousands of projects by examining the source code, but not their concrete software engineering principles. Instead, \autoref{sec:results} will show particular tendencies for specific programming languages, which are then discussed based on the formulated research questions in \autoref{sec:discussion}.
