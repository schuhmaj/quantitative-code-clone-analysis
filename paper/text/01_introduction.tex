% !TeX spellcheck = en_US

\section{Introduction}
\label{sec:intro}

What exactly does the word \textit{quality} mean? The Oxford Dictionary defines it as ``the standard of something as measured against other things of a similar kind''.\footnote{\url{https://arcadia.io/the-final-word-quality/} last accessed: 15.07.2022}, which only leads to more questions about the meaning of the words: standard, measurement, similar kind. A pretty tricky definition, although the word can be found nearly everywhere.
If one zooms in, one can try to partially define the word, e.g., in the context of software engineering. According to the ISO 9000 - Quality management, ``(Software) Quality is the degree to which a set of inherent characteristics fulfills requirements, meaning needs or expectations that are stated, generally implied or obligatory'' \cite{matthes2020ase}.
Even this restricted definition is still a bit fuzzy. By further zooming in, one can separate between internal and external quality. Here, the former describes the properties of the code in the context of a software system \cite{pretschner2022requirements}, whereas the latter formulates the user-observable criteria.
In the following, we will stay with the former and focus on the criterion clone coverage specifically.
This property is defined as the density of code clones and describes the relative number of statements being part of a clone, i.e., duplicated \cite{knilling2020priorisierung}.
The phenomenon of code clones is a relatively common issue in software projects. Previous studies showed that software projects typically contain $7 -23\%$ code duplication \cite{koschke2007survey}.
However, there is no absolute consensus about the harmfulness of code clones, similar to the fuzzy definition of quality. As Kasper et al. \cite{kasper2006cloning} state, cloning can be an efficient engineering tool for testing new features, i.e., quick prototyping.
On the other hand, code duplicates can ``increase maintenance costs [and] can create faults'' \cite{juergens2009code} due to inconsistent changes to them, which harm consequently correctness and, thereby, quality.
Generally, this work does not aim to answer how code clones affect quality for good or worse - the symptoms - but instead tries to exemplify which circumstances lead to their occurrences with a specific remark on quantitatively studying the consequences of programming language choice.