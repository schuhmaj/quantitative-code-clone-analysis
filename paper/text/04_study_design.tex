% !TeX spellcheck = en_US

\section{Study Design}
\label{sec:study_design}

In this section, the methodology and analysis process of the results of \autoref{sec:results} is presented. First, the overview is given before than looking at a more detailed description.


\subsection{Overview}

\begin{figure}[tbh]
	\centering
	\includegraphics[width=\linewidth]{figures/setup/Code-Clone-Analysis-Deployment}
	\caption{UML Deployment Diagram of the presented examination}
	\label{fig:overview_deployment}
\end{figure}

In this project, we generally analyzed $5224$ projects in seven different programming languages: pure \texttt{C}, \texttt{C/C++}, \texttt{Java}, \texttt{Kotlin}, \texttt{Rust}, \texttt{Python} and \texttt{Go}. Based on the asked \acp{rq} the former programming languages are a logical consequence to include. The latter two are chosen for comparability reasons. \texttt{Go} is comparably new exactly like \texttt{Kotlin} and \texttt{Rust} and has similar use cases. On the other hand, \texttt{Python} is, as of now, one of the most popular programming languages and widely used \cite{stackoverflow2021languages}.

All of these projects are located on \textit{GitHub} and extracted from the corresponding \textit{awesome list}\awesomeFootnote{}. Those lists contain a curated list of recommendations for each programming language including frameworks and libraries for all kind of utility like testing, logging, etc.
The lists are equal in content and structure, but not in size. Thereby, the $5224$ projects are not uniformly distributed, but rather some languages have a greater share. However, the analysis concentrates on the mean and standard deviation of the clone coverage (with respect to \ac{sloc}). Therewith, it is independent from the number of projects, but rather requires comparability in terms of the actual projects' content.

The analysis core relies on \teamscale{}\teamscaleFootnote{} for the examination of each project. \teamscale{} is a "Software Intelligence Platform" capable of analyzing and monitoring the properties of a software project including clone detection in a great variety of programming languages. The utilized version \texttt{8.0.5} provides a REST API through which these capabilities are accessible. The here presented toolchain uses this circumstance.

The interaction of \teamscale{} with the two of the four components of the Python script\gitFootnote{} can be seen in \autoref{fig:overview_deployment}. \autoref{sec:implementation} describes the Python script's components in more detail.

\subsection{Implementation}
\label{sec:implementation}

The Python script consists of four components which need to be executed independently after one another. The whole procedure is demonstrated in \autoref{fig:overview_communication}.

\begin{figure}[tbh]
	\centering
	\includegraphics[width=\linewidth]{figures/setup/Code-Clone-Analysis-Communication}
	\caption{Diagram showing every processing step in order and associated data flow}
	\label{fig:overview_communication}
\end{figure}

\subsubsection{Project Extractor}

The first components generally fetches the given \textit{awesome list} for programming language $X$ and extracts the \textit{GitHub} Repositories' Links. Further in validates, the existence of those repositories and finds out the latest revision and the name of the repositories default branch. The results \texttt{Project} objects are then serialized and stored in a \texttt{pickle} file which acts as persistent memory (compare \autoref{fig:overview_communication}).

\subsubsection{Teamscale Administrator}

The second component reads de-serializes the pickled file again to members of \texttt{Project} and creates these projects in \teamscale|{} with a \textit{GitConnector}\footnote{\url{https://docs.teamscale.com/reference/connector-options/}, last accessed: 11.07.2022}. Generally, this has been executed bit by bit with a batch size of $100$ \texttt{Projects} each to not overload the computer hosting \teamscale{}.

\subsubsection{Teamscale Extractor}

After \teamscale{} has finalized the analysis, the third component fetches the projects one by one with their respective ID and appends the analysis data to the \texttt{Project}. Again, the data is written to the persistent memory.

\subsubsection{Project Analyzer}

The last component reads the fully annotated projects and plots and statistically analyzes the results. Its results is presented in the \autoref{sec:results}.