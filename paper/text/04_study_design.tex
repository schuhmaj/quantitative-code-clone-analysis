% !TeX spellcheck = en_US

\section{Study Design}
\label{sec:study_design}

In this section, the methodology behind the performed empirical study is presented. First, an overview is provided before looking at a more thorough description.

\subsection{Overview}

\begin{figure}[tbh]
	\centering
	\includegraphics[width=\linewidth]{figures/setup/Code-Clone-Analysis-Deployment}
	\caption{UML Deployment Diagram of the presented analysis toolchain}
	\label{fig:overview_deployment}
\end{figure}

In this project, we analyzed $5224$ projects in seven different programming languages with a custom Python script\gitFootnote{}: pure \texttt{C}, \texttt{C/C++}, \texttt{Java}, \texttt{Kotlin}, \texttt{Rust}, \texttt{Python}, and \texttt{Go}. Including the first five programming languages emerges from \refrq{question:comparing_age} and its sub-questions. The latter two are chosen for two reasons: \texttt{Go} is comparably new, exactly like \texttt{Kotlin} and \texttt{Rust}, and has similar use cases. Secondly, as of now, \texttt{Python} is one of the most popular programming languages and is widely used \cite{stackoverflow2021languages}.

All of the analyzed projects are located on \textit{GitHub} and extracted from the corresponding \textit{awesome list}\awesomeFootnote{}. Those lists contain a curated list of community-based recommendations for each programming language, including frameworks and libraries for all kinds of utilities like testing, logging, and more.
The lists are equal in content and structure but not in size. Thereby, the $5224$ projects are not uniformly distributed; rather, some languages have a greater share. However, the analysis concentrates on the mean and standard deviation of the clone coverage. Therefore, it is independent of the number of projects but requires a significant quantity of projects for expressive numbers.

The analysis core relies on \teamscale{}\teamscaleFootnote{} for the examination of each project. \teamscale{} is a ``Software Intelligence Platform'' capable of analyzing and monitoring the quality metrics of a software project, including clone detection in various programming languages. The utilized version \texttt{8.0.5} provides a REST API through which these capabilities are accessible. The toolchain presented in the following uses this possibility.

The interaction of \teamscale{} with two of the four components of the Python script used for evaluation can be seen in \autoref{fig:overview_deployment}. \autoref{sec:implementation} describes the Python script's components in more detail.

\subsection{Implementation}
\label{sec:implementation}

The Python script\gitFootnote{} consists of four components that need to be executed independently after one another. The entire procedure is documented in \autoref{fig:overview_communication}.

\begin{figure}[tbh]
	\centering
	\includegraphics[width=\linewidth]{figures/setup/Code-Clone-Analysis-Communication}
	\caption{Diagram showing every processing step in order and associated data flow}
	\label{fig:overview_communication}
\end{figure}

\subsubsection{Project Extractor}

The first component fetches the given \textit{awesome list} for programming language $X$ and extracts the \textit{GitHub} repositories’ links. Further, it validates the existence of those repositories and finds the latest revision and the name of the default branch. The results \texttt{Project} objects are then serialized and stored in a \texttt{pickle} file which acts as persistent memory (compare \autoref{fig:overview_communication}).

\subsubsection{Teamscale Administrator}

The second component deserializes the pickled file again to members of \texttt{Project} and creates these projects in \teamscale|{} with a \textit{GitConnector}\footnote{\url{https://docs.teamscale.com/reference/connector-options/}, last accessed: 11.07.2022}. Generally, this has been executed with a batch size of $100$ \texttt{Projects} each to prevent overloading the computer hosting \teamscale{}.

\subsubsection{Teamscale Extractor}

After \teamscale{} has finalized the analysis, the third component fetches the projects one by one with their respective ID and appends the analysis data to the \texttt{Project}. Again, the data is written to the persistent memory.

\subsubsection{Project Analyzer}

The last component reads the fully annotated projects and plots and statistically analyzes the results by using Welch's $t$-test and Permutation Testing. (Its results are presented in the \autoref{sec:results}.)