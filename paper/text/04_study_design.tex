% !TeX spellcheck = en_US

\section{Study Design}
\label{sec:study_design}

In this section, the methodology and analysis process of the results of \autoref{sec:results} is presented. First, the overview is given before than looking at a more detailed description.


\subsection{Overview}

\begin{figure}[htb]
	\centering
	\includegraphics[width=\linewidth]{figures/setup/Code-Clone-Analysis-Deployment}
	\caption{UML Deployment Diagram of the presented examination}
	\label{fig:overview_deployment}
\end{figure}

In this project, we generally analyzed $5224$ projects in seven different programming languages: pure \texttt{C}, \texttt{C/C++}, \texttt{Java}, \texttt{Kotlin}, \texttt{Rust}, \texttt{Python} and \texttt{Go}. These projects are all located on \textit{GitHub} and extracted from the corresponding \textit{awesome list}\awesomeFootnote{}. Those lists contain a curated list of recommendations for each programming language including frameworks and libraries for all kind of utility like testing, logging, etc.
The lists are equal in content and structure, but not in size. Thereby, the $5224$ projects are not uniformly distributed, but rather some languages have a greater share. However, the analysis concentrates on the mean and standard deviation of the clone coverage (with respect to \ac{sloc}). Therewith, it is independent from the number of projects.

The analysis core relies on \teamscale{}\teamscaleFootnote{} for the examination of each project. \teamscale{} is a "Software Intelligence Platform" capable of analyzing and monitoring the properties of a software project including clone detection in a great variety of programming languages. The utilized version \texttt{8.0.5} provides a REST API through which these capabilities are accessible. The here presented toolchain uses this circumstance.

The interaction of \teamscale{} with the two of the four components of the python script can be seen in \autoref{fig:overview_deployment}. \autoref{sec:implementation} presents those components in more detail.

\subsection{Implementation}
\label{sec:implementation}

\begin{figure}[htb]
	\centering
	\includegraphics[width=\linewidth]{figures/setup/Code-Clone-Analysis-Communication}
	\caption{Diagram showing every processing step in order and associated data flow}
	\label{fig:overview_communication}
\end{figure}