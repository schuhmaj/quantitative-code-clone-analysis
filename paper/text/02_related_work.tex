% !TeX spellcheck = en_US

\section{Related Work}
\label{sec:related_work}

In general, we can subdivide this section into two categories. First, existing works searching for the reasons for code clones in human, organizational, and technological factors are examined, and secondly, we focus on papers describing the correlation between specific languages and the appearance of code clones explicitly.

\subsection{Reasons for Code Clones}
\label{sec:reasons_for_code_clones}

\subsubsection{Human \& Organizational Factors}

Before describing how the choice of programming language can affect clones, we start with a brief overview of other reasons for clones. Here, first to mention are human factors like inadvertently or impatiently copying because time or understanding are missing. Missing abstraction and the need to fulfill particular coupling and cohesion properties are also potential sources of generating code clones  \cite{kasper2006cloning}.
The intention to minimize the risks when adopting new ideas in software is also a driver for cloning code since this technique allows to keep errors through the introduced redundancy in just a single module  \cite{cordy2003comprehending}. Further, as time-to-market is critical, cloning can improve the speed of developing an early prototype, as analyzed by \cite{rajapakse2007using} for web applications.

\subsubsection{Technological Factors}

Alongside these human and economic factors stand "[t]echnology limitations" \cite{kasper2006cloning} like using a specific programming language.
\textit{Templating} is a powerful feature of modern programming languages like \texttt{C++}. In contrast, languages like \texttt{C} do not offer any equivalent feature.
Given a certain algorithm working with double precision in the \texttt{C} programming language, the developer is forced to copy the procedure and replace \texttt{double} with \texttt{float}. Such "boiler-plating [is enforced only] due to language inexpressiveness" \cite{kasper2006cloning}. 
Besides those issues with \textit{Templating}, i.e., generalization, in specific languages exist deadlocked ways and patterns how to fulfill specific tasks.
For example, creating a \textit{Socket} and communicating with it varies and is different in each programming language - sometimes shorter, sometimes more lengthy, but usual there is one optimal way of doing it so that possible exception (if the language even has such a feature) or error codes are handled. These optimal sequences will then be often copy-and-pasted by developers \cite{kasper2006cloning}.
Further to \textit{Templating}, Kasper et al.  \cite{kasper2006cloning} mention \textit{customization} as a reason for duplication.
For instance, code ownership can make bug fixing difficult, only allowing the creation of a work-around by copying and improving the faulty lines.
Further, they enumerate the idiom of \textit{replicate and specialize}, in which developers clone code to specialize a solution rather than generalizing an existing implementation.
Lastly, Kasper et al. \cite{kasper2006cloning} describe \textit{exact matches}, in the context of code clones, as a result of copying ”semantic properties [of] otherwise unrelated functionality” \cite{kasper2006cloning} between methods like logging or debugging statements and reusing characteristic code structures like loops which are easier copied than implemented as a reusable function.

\subsection{Similar Analyses}
\label{sec:similiar_analyses}

This work will later present a comparison of clone coverage in different generation programming languages, but there are already some studies with a similar scope.

\subsubsection{Code Clones in Build Systems}

Once developer produced the different components of a large-scale software system, they typically need to be assembled together. This critical step usually accomplished with certain build systems like \texttt{Ant}, \texttt{Maven}, \texttt{Autotools} or \texttt{CMake} for respectively \texttt{Java} and \texttt{C/C++}. \cite{mcintosh2014collecting} compares these build systems with surprising results. As overall result, "build systems tend to have higher cloning rates than other software artifacts" \cite{mcintosh2014collecting}. 
They conclude with the finding that, as opposed of what one might think, modern build systems like \texttt{CMake} and \texttt{Maven} have higher clone coverage then their older counterparts and the \texttt{C/C++} systems and vice-versa \texttt{Java} build systems have a higher clone coverage, in many cases more than $50\%$, than the respective \texttt{C/C++} counterparts.

\subsubsection{Comparison of Java \& Scala}

Jorge et. al. \cite{jorge2012impact} directly compares features of the two languages \texttt{Java} and \texttt{Scala} and studies how their language constructs correlate to code cloning. In their findings, they conclude that code duplication problems arise with higher probability if a language is more verbose, e.g. due to getters and setters, anonymous classes, constructors, and lacks (simple) abstraction capabilities. Properties which finally lead to more effort refactoring code than simply cloning a source code. \cite{jorge2012impact}
